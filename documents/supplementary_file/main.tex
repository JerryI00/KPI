\documentclass[11pt]{article}

\usepackage{fullpage}
\usepackage[ruled, linesnumbered, vlined, commentsnumbered]{algorithm2e}
\usepackage{graphicx,subfig} % figure related
\usepackage{amsfonts,amssymb,amsmath,amsthm,amsopn}	% math related
\usepackage{hhline,booktabs,colortbl,diagbox,multirow,tabularx,threeparttable} % table related

\usepackage{enumerate}
\usepackage{authblk}
\usepackage{footnote}
\usepackage{hyperref}
\usepackage{prettyref}
\usepackage{cite}
\usepackage{setspace}
\usepackage{color}
\usepackage{xcolor}  % Required for custom colors
\usepackage{scrpage2}
\usepackage{geometry}

% tikz pakcages
\usepackage{tikz}
\usepackage{pgfplots}
\usetikzlibrary{positioning,shapes,shadows,arrows,calc}
\tikzstyle{component}=[rectangle, draw=black, rounded corners, fill=blue!40, drop shadow, text centered, anchor=north, text=white, minimum height=1cm]
\tikzstyle{arrow}=[->, thick]

\pgfplotsset{compat=1.12}
\usetikzlibrary{intersections}
\usetikzlibrary{pgfplots.statistics}
\usepgfplotslibrary{fillbetween}

\geometry{a4paper,left=2cm,right=2cm,top=2cm,bottom=2cm} % set the page layout

% Define a few colors for making text stand out within the presentation
\definecolor{myblue}{RGB}{34,31,217}
\definecolor{mycyan}{gray}{.7}
\definecolor{Gray}{gray}{0.9}

\newtheorem{remark}{Remark}
\newtheorem{theorem}{Theorem}
\newtheorem{proposition}{Proposition}
\newtheorem{corollary}{Corollary}
\newtheorem{definition}{Definition}
\newtheorem{lemma}{Lemma}
\newtheorem{property}{Property}

\DeclareMathOperator*{\argmax}{argmax}
\DeclareMathOperator*{\argmin}{argmin}

% correct bad hyphenation here
\hyphenation{op-tical net-works semi-conduc-tor}

\newcommand{\bb}[1]{\multicolumn{1}{>{\columncolor{mycyan}}c}{\textbf{{#1}}}}
\newcommand\notealf[1]{\mbox{}\marginpar{\footnotesize\raggedright\hspace{0pt}\color{blue}\emph{#1}}}
\newcommand{\pref}{\prettyref}

\newrefformat{fig}{Fig.~\ref{#1}}
\newrefformat{tab}{Table~\ref{#1}}
\newrefformat{sec}{Section~\ref{#1}}
\newrefformat{app}{Appendix~\ref{#1}}
\newrefformat{alg}{Algorithm~\ref{#1}}
\newrefformat{property}{Property~\ref{#1}}
\newrefformat{theorem}{Theorem~\ref{#1}}
\newrefformat{corollary}{Corollary~\ref{#1}}
\newrefformat{proposition}{Proposition~\ref{#1}}
\newrefformat{def}{Definition~\ref{#1}}
\newrefformat{eq}{equation~(\ref{#1})}

\renewcommand\qedsymbol{$\blacksquare$}
\renewcommand\Authands{ and }

\usepackage{lscape}

\newcommand*{\email}[1]{%
    \normalsize\href{mailto:#1}{#1}\par
    }

\begin{document}

%% title
\title{\vspace{-1ex}\LARGE\textbf{Supplementary Document of \lq\lq Knee Point Identification Based on Trade-Off Utility\rq\rq}\footnote{This manuscript is submitted for potential publication. Reviewers can use this manuscript in their referee.}}

%% authors and affiliations
\author[1]{\normalsize Ke Li$^{\#}$}
\author[2]{\normalsize Haifeng Nie$^{\#}$}
\author[2]{\normalsize Huiru Gao}
\author[1]{\normalsize Geyong Min} 
\author[3,4]{\normalsize Xin Yao}
\affil[1]{\normalsize Department of Computer Science, University of Exeter, EX4 4QF, Exeter, UK}
\affil[2]{\normalsize College of Computer Science \& Engineering, University of Electronic Science and Technology of China, Chengdu 611731, China}
\affil[3]{\normalsize Department of Computer Science and Engineering, Southern University of Science and Technology, Shenzhen, China}
\affil[4]{\normalsize CERCIA, School of Computer Science, University of Birmingham, B15 2TT, Birmingham, UK}
\affil[$\ast$]{\normalsize Email: \texttt{\{k.li, g.fu\}@exeter.ac.uk, rxc332@cs.bham.ac.uk, xiny@sustc.edu.cn}}
\affil[$\#$]{\normalsize The first two authors make equal contributions to this paper.}

\date{}
\maketitle

% Table generated by Excel2LaTeX from sheet 'Sheet1'
\begin{table}[htbp]
  \centering
  \caption{Parameter Settings for all test problems}
    \begin{tabular}{c|c|c|c|c|c}
    \hline
    Test problem & A\protect\footnotemark[3]    & s     & B    & l  \\
    \hline
    DO2DK & 1, 2 & 1   &   & 0 \\
    \hline
    CKP & 1, 2 & & & &\\
    \hline
    DEB2D & 1, 2 &  &  &  &  \\
    \hline
    DEB3D & 1, 2 &  &  &  &  \\
    \hline
    PMOP1 & 2, 4 & -2    & 1      & 0 \\
    \hline
    PMOP2 & 2, 4 & -1    & 12, 1  & 0 \\
    \hline
    PMOP3 & 2, 4 & 1     & 1      & 0 \\
    \hline
    PMOP4 & 2, 5 & -3    & 1      & 0 \\
    \hline
    PMOP5 & 1  & -0.5  & 1      & 12 \\
    \hline
    PMOP6 & 2, 3 & 2     & 1      & 0 \\
    \hline
    PMOP7 & 1, 3 & 0, 1  & 1      & 0 \\
    \hline
    PMOP8 & 2, 4 & -2    & 1      & 0 \\
    \hline
    PMOP9 & 2, 4 & 1     & 1      & 0 \\
    \hline
    PMOP10 & 1  & -0.5  & 1     & 6 \\
    \hline
    PMOP11 & 2, 4 & 0     & 1     & 0 \\
    \hline
    PMOP12 & 2, 4 & 2     & 1     & 0 \\
    \hline
    PMOP13 & 1, 3 & -2    & 1     & 0 \\
    \hline
    PMOP14 & 2, 4 & -1    & 1     & 0 \\
    \hline
    \end{tabular}%
  \label{tab:addlabel}%
\end{table}%
\protect\footnotetext[3]{$A$ is an integer to control the number of knee points. $B$ controls the location of the knee points. Parameter $s > 0$ will skew the PoF}

\section*{Acknowledgment}
K. Li was supported by UKRI Future Leaders Fellowship (Grant No. MR/S017062/1) and Royal Society (Grant No. IEC/NSFC/170243).   X. Yao was supported by EPSRC (Grant No. EP/P005578/1), the Program for Guangdong Introducing Innovative and Enterpreneurial Teams (Grant No. 2017ZT07X386), Shenzhen Peacock Plan (Grant No. KQTD2016112514355531) and the Program for University Key Laboratory of Guangdong Province (Grant No. 2017KSYS008). 

\bibliographystyle{IEEEtran}
\bibliography{IEEEabrv,knee}

\end{document}
